\documentclass[12pt]{article}

\usepackage{amsmath}
\usepackage{amssymb}
\usepackage{graphicx}
\usepackage{hyperref}
\usepackage{algorithm}
\usepackage{tikz}
\usepackage{pgfplots}
\usepackage{algpseudocode}
\usepackage[margin=1in]{geometry}
\pgfplotsset{compat=1.18}

\newcommand{\examtitle}{SEMESTER YEAR - Midterm NUMBER}
\newcommand{\examAuthor}{Prof. Vibhab Gogate}

\title{\examtitle~Solutions}
\author{UTD ECS et al.}

\begin{document}

\maketitle

\section*{Question 1}
\subsection*{Problem Statement}
Write a recursive algorithm to compute the $n$-th Fibonacci number. The Fibonacci sequence is defined as follows:

\subsection*{Solution}
\begin{algorithm}
    \caption{Fibonacci Sequence}
    \begin{algorithmic}[1]
    \Procedure{Fib}{n}
        \If{$n \leq 1$}
            \State \Return $n$
        \Else
            \State \Return \Call{Fib}{$n-1$} + \Call{Fib}{$n-2$}
        \EndIf
    \EndProcedure
    \end{algorithmic}
\end{algorithm}

\section*{Question 2}
\subsection*{Problem Statement}
For each of the following datasets where “◦” represents negative examples and “×” represents positive examples, write alongside each classifier whether it will have zero training error on the dataset. Also, explain why in one sentence or by drawing a decision boundary. No credit if the explanation is incorrect.

\subsection*{Solution}

\begin{tikzpicture}
\begin{axis}[
    xmin=0, xmax=5.5,
    ymin=0, ymax=5.5,
    axis lines=middle,
    xtick={0,1,2,3,4,5},
    ytick={0,1,2,3,4,5},
    xlabel={$x$},
    ylabel={$y$},
    width=10cm,
    height=10cm,
    enlargelimits={abs=0.5},
]

% Plotting circles (o)
\addplot[only marks, mark=o, mark size=3pt] coordinates {
    (2,2) (2.5,2) (2,2.5) (2.5,2.5) (2.2,2.3)
};

% Plotting crosses (x)
\addplot[only marks, mark=x, mark size=5pt] coordinates {
    (3,3) (3.5,3.5) (4,3) (4.5,4) (3.5,4.2)
};

% Drawing red line from (2,4) to (1,5)
\addplot[red, thick] coordinates {(1,4.5) (5,1)};

\end{axis}
\end{tikzpicture}


\section*{Credits}
These questions are taken from the \textit{\examtitle} of the Machine Learning course at the University of Texas at Dallas, and \examAuthor~in particular. The provided solutions are a collaborative effort of the students.

\end{document}